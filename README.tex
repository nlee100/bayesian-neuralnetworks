\documentclass[]{tufte-handout}

% ams
\usepackage{amssymb,amsmath}

\usepackage{ifxetex,ifluatex}
\usepackage{fixltx2e} % provides \textsubscript
\ifnum 0\ifxetex 1\fi\ifluatex 1\fi=0 % if pdftex
  \usepackage[T1]{fontenc}
  \usepackage[utf8]{inputenc}
\else % if luatex or xelatex
  \makeatletter
  \@ifpackageloaded{fontspec}{}{\usepackage{fontspec}}
  \makeatother
  \defaultfontfeatures{Ligatures=TeX,Scale=MatchLowercase}
  \makeatletter
  \@ifpackageloaded{soul}{
     \renewcommand\allcapsspacing[1]{{\addfontfeature{LetterSpace=15}#1}}
     \renewcommand\smallcapsspacing[1]{{\addfontfeature{LetterSpace=10}#1}}
   }{}
  \makeatother

\fi

% graphix
\usepackage{graphicx}
\setkeys{Gin}{width=\linewidth,totalheight=\textheight,keepaspectratio}

% booktabs
\usepackage{booktabs}

% url
\usepackage{url}

% hyperref
\usepackage{hyperref}

% units.
\usepackage{units}


\setcounter{secnumdepth}{-1}

% citations

\newlength{\cslhangindent}
\setlength{\cslhangindent}{1.5em}
% For Pandoc 2.8 to 2.11
\newenvironment{cslreferences}%
  {}%
  {\par}
% For pandoc 2.11+ using new --citeproc
\newlength{\csllabelwidth}
\setlength{\csllabelwidth}{3em}
\newlength{\cslentryspacingunit} % times entry-spacing
\setlength{\cslentryspacingunit}{\parskip}
\newenvironment{CSLReferences}[2] % #1 hanging-ident, #2 entry spacing
 {% don't indent paragraphs
  \setlength{\parindent}{0pt}
  % turn on hanging indent if param 1 is 1
  \ifodd #1
  \let\oldpar\par
  \def\par{\hangindent=\cslhangindent\oldpar}
  \fi
  % set entry spacing
  \setlength{\parskip}{#2\cslentryspacingunit}
 }%
 {}
\usepackage{calc}
\newcommand{\CSLBlock}[1]{#1\hfill\break}
\newcommand{\CSLLeftMargin}[1]{\parbox[t]{\csllabelwidth}{#1}}
\newcommand{\CSLRightInline}[1]{\parbox[t]{\linewidth - \csllabelwidth}{#1}}
\newcommand{\CSLIndent}[1]{\hspace{\cslhangindent}#1}

% pandoc syntax highlighting

% table with pandoc

% multiplecol
\usepackage{multicol}

% strikeout
\usepackage[normalem]{ulem}

% morefloats
\usepackage{morefloats}


% tightlist macro required by pandoc >= 1.14
\providecommand{\tightlist}{%
  \setlength{\itemsep}{0pt}\setlength{\parskip}{0pt}}

% title / author / date
\title{Bayesian Learning Artificial Neural Networks for Modeling
Survival Data}
\author{Amos Okutse}
\date{}

\usepackage{booktabs}
\usepackage{longtable}
\usepackage{array}
\usepackage{multirow}
\usepackage{wrapfig}
\usepackage{float}
\usepackage{colortbl}
\usepackage{pdflscape}
\usepackage{tabu}
\usepackage{threeparttable}
\usepackage{threeparttablex}
\usepackage[normalem]{ulem}
\usepackage{makecell}
\usepackage{xcolor}

\begin{document}

\maketitle




Amos Okutse, Naomi Lee

29 April, 2022

\begin{quote}
\hypertarget{abstract}{%
\subsection{Abstract:}\label{abstract}}

This checking whether the abstract renders correctly!
\end{quote}

Text from this notebook has been adopted from:

\hypertarget{premise}{%
\section{Premise}\label{premise}}

\hypertarget{some-cat}{%
\subsection{Some cat!}\label{some-cat}}

\hypertarget{alternatively}{%
\subsection{Alternatively\ldots{}}\label{alternatively}}

\#\includegraphics{https://www.fairobserver.com/wp-content/uploads/2021/07/artificial-intelligence.jpg}
Image Laurent T/Shutterstock

The Bayesian paper in BDNN
(\protect\hyperlink{ref-feng2021bdnnsurv}{Feng and Zhao 2021})
paranthetical citation

BLNN (\protect\hyperlink{ref-sharaf2020blnn}{Sharaf et al. 2020})

\hypertarget{references}{%
\section{References}\label{references}}

\hypertarget{refs}{}
\begin{CSLReferences}{1}{0}
\leavevmode\vadjust pre{\hypertarget{ref-feng2021bdnnsurv}{}}%
Feng, Dai, and Lili Zhao. 2021. {``BDNNSurv: Bayesian Deep Neural
Networks for Survival Analysis Using Pseudo Values.''} \emph{arXiv
Preprint arXiv:2101.03170}.

\leavevmode\vadjust pre{\hypertarget{ref-sharaf2020blnn}{}}%
Sharaf, Taysseer, Theren Williams, Abdallah Chehade, and Keshav Pokhrel.
2020. {``BLNN: An r Package for Training Neural Networks Using Bayesian
Inference.''} \emph{SoftwareX} 11: 100432.

\end{CSLReferences}



\end{document}
